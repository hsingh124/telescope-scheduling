\documentclass{article}
\usepackage[utf8]{inputenc}
\usepackage{marginnote}
\usepackage{biblatex}
\addbibresource{References.bib}
\usepackage{geometry}
\usepackage{hyperref}
\hypersetup{
    colorlinks=true,
    linkcolor=blue,
    filecolor=blue,      
    urlcolor=blue,
    citecolor=blue
}
\geometry{
    a4paper,
    total={170mm,257mm},
    top=20mm,
}
\setlength\parindent{0pt}
\setlength{\parskip}{0.5em}

\title{Research notes on Telescope Scheduling}
\author{Harpreet Singh}
\date{2021}

\begin{document}

\maketitle

\newgeometry{
    left=11mm,
    top=20mm,
}

\section*{\cite{bellm2019zwicky} \citetitle*{bellm2019zwicky}}

\subsection*{Introduction and ZTF}

ZTF: \url{https://www.ztf.caltech.edu/}

In this work we consider the specific scheduling problem of
a single-telescope ground-based wide-field imaging survey. We
are focused on its application to the Zwicky Transient Facility
project, which imposes some specific requirements, but our
formalism is relevant for other time-domain surveys, such as
those conducted with the LSST, the Dark Energy Camera, 
and Hyper Suprime-cam. Minor modifications would enable its
use by multi-telescope surveys such as the Asteroid Terrestrialimpact
Last Alert System, PanSTARRS, the All-sky Automated Survey for
Supernovae, and BlackGEM.

Simply stated, the scheduling problem to be solved is to
determine which fields to observe in what order, with a goal of
maximizing an objective function while achieving the desired
temporal spacing of observations (“cadence”). Optimizing the
survey schedule provides a greater quantity of high-quality
data, increasing the scientific output of the survey.

\textbf{Python Implementation: \url{https://github.com/ZwickyTransientFacility/ztf_sim}}

\subsection*{Constraints}
This weighting combines in a self-consistent way many
factors that are intuitively relevant for assessing whether an
image is “good”: the limiting magnitude depends on the filter,
seeing, airmass, and sky brightness. We use a model
to predict the variation in limiting magnitude and
hence our metric as a function of these time-varying inputs.
Accordingly, our optimization will naturally select exposures
near zenith and away from the moon; but by combining them in
a single scalar the optimization can coherently trade these
factors against one another as they change through the night.

Our metric deliberately does not contain factors that account
for relative scientific priority or cadence. These concerns have
no general quantitative relationship to our objective function or
each other. Instead, we use the structure of the optimization
algorithm to impose these constraints.

Our optimization algorithm maximizes the
summed metric over an entire night. In cases where a greedy
algorithm is more convenient, it is simple to define an
instantaneous volumetric survey speed, \[V \propto 10^{0.6m_{lim}}/(t_{exp}+t_{OH})\]
that normalizes the volume probed in an exposure by the time
required to obtain it, a sum of the exposure time \(t_{exp}\) and any
readout or slew overheads \(t_{OH}\).

\subsection*{Problem}

The ZTF scheduler attempts to maximize the total
number of exposures taken per night, weighted by the spatial
volume probed by each, and subject to the constraints imposed by
program balance and cadence. If the observing
cadences are well chosen, maximizing this quantity will maximize
the transient discovery rate. Bellm explores the relationship
between the chosen observing cadences, a survey’s volumetric
and areal survey rates, and the transient detection rate.

Neglecting cosmological effects, the volume \(V_{lim}\) probed by
a given exposure is proportional to the cube of the limiting
distance \(d_{lim}\) a transient of fiducial absolute magnitude M can
be detected given the limiting magnitude \(m_{lim}: V_{lim} \propto d_{lim}^3\),
where \(d = 10^{0.2(m_{lim}-M+5)}\)pc. The volumetric
weighting per exposure is thus \[V = 10^{0.6(m_{lim} - 21)},\]
where we have absorbed constant factors and normalized to a
convenient limiting magnitude for ZTF.

We construct the observing schedule by dividing the night
into a set of temporal blocks T. The set of available filters in the camera is F. The set of
Request Sets from all observing programs P is R.

\(Y_{rtf}\) = 1 if Request Set \(r \in R\) has an observation scheduled
at time block \(t \in T\) using filter \(f \in F\), and 0 otherwise

\(Y_s\) = 1 if the filter changes between time blocks \(s \in T\)
and \(s + 1 \in T\), and 0 otherwise.

The optimizer maximizes an objective function which sums
the volume-weighted (formula specified in the equation above) number of exposures
scheduled through the night. Because of how we constrain the
number of exposures in a temporal block, we also
penalize for exposures lost due to filter changes. The objective
function is thus
\[max\left(\left(\sum_{r \in R}\sum_{t \in T}\sum_{f \in F}V_{rtf}Y_{rtf}) - (\frac{t_{filt}}{t_{exp}+t_{OH}}w\sum_{t \in T}Y_s\right)\right)\]

where \(t_{filt}\) is the time required to change filters and w is a weight
factor \((\approx(V_{rtf}))\) accounting for the value of each lost
exposure.

\section*{\cite{moser2018dispatch} \citetitle*{moser2018dispatch}}

\marginnote{This paper contains a lot of really useful information from 
other papers. Some points from this paper are listed in the 
\textit{Extra notes} section below. 
\textbf{Definately give this another read}}[1cm]

\subsection*{Introduction and related work}

We consider strategies
for minimising the time required to observe a fixed set of pulsars, which provides
an excellent example of scheduling in an unpredictable environment. First, owing to
scintillation, the intensity of a pulsar signal is variable and random; therefore, the
decision to abort or prolong an observation can be made only after some fraction of
the scheduled observation has been completed. Second, observations may be interrupted by 
radio frequency interference or when the source sets below the horizon.
Some sources are visible for more or less time depending on their declination and the
latitude of the observing telescope. Formulating the problem in these terms leads to
a highly dynamic shortest path problem with uncertainty. Unlike other documented
telescope scheduling approaches, we demonstrate how a simple earliest setting policy
achieves sets of pulsar observations in a rather short timespan.

Telescope arrays have motivated the development of a new generation of scheduling algorithms. 
ALMA, an array of 66 telescopes, has been operating since 2012,
while the construction of the Square Kilometre Array (SKA) is planned to begin in
2020. The SKA has motivated the Master’s thesis by Buchner \cite{buchner2011dynamic}, who proposed a
number of scheduling policies, such as choosing the task with the highest priority, the
task with the smallest observation time, the task that continues from the preceding
time slot, and compared them with the performance of a Genetic Algorithm (GA). A
Mixed-Integer Linear Program was also applied but abandoned for excessive running
time. Buchner’s objective function maximises the observation time weighted by the
task priorities.

Our study is also motivated by the SKA and the optimal scheduling of large scale
pulsar surveys and pulsar monitoring programs that will be undertaken with the next
generation of radio telescopes. In this paper, we focus on the problem of monitoring a
large number of pulsars in a dynamic environment. 

\subsection*{Constraints, variables and assumptions}

Two physically-motivated effects
are introduced to bring variability to the scheduling problem:
\begin{enumerate}
    \item Pulsar scintillation impacts on the sensitivity of the observation and may lead to
    either an extension of the integration time or premature abortion. Scintillation
    is random and the decision to continue, extend or abort can be made only after
    some fraction of the scheduled observation has been completed.
    \item  Interference by satellites can render observations useless and, in extreme cases,
    damage observatory equipment. Given up-to-date ephemerides, satellite positions
    can be predicted and observations that would be corrupted by them can be avoided.
\end{enumerate}

Although we focus on pulsars, these two classes of dynamic environmental 
influences (random and predictable) are quite generic and applicable to 
other astronomical scheduling problems.

In addition to the introduction of a dynamic environment, several aspects of our
work differ from previous studies (such as that by Buchner):

\begin{itemize}
    \item  All approved observations have to be carried out, hence the priorities of
    observations do not contribute to the objective function.
    \item Continuous time is assumed rather than slots.
    \item The initial formulation assumes a single telescope.
    \item  The schedule is simulated in conditions modelled on actual 
    observation conditions to obtain a likely outcome given the uncertainty 
    of the problem.
\end{itemize}

Consequently, we have chosen to study the observation of a set of pulsars as an
example of a highly dynamic telescope scheduling problem with uncertainty.

\subsection*{Problem and objectives}
In astronomy, observing programmes can be broadly classified as targeted observations of known sources and systematic scans of areas of the sky, such as surveys that
aim to map a large-scale structure or discover new objects.

Astronomers submit proposals for observations to telescope operators who
approve proposals based on their merit and allocate telescope time. The proposals
detail the level of priority of a target, its location and desired observation time. Generally, proposals are assessed biannually, to be processed over the subsequent 6 months.
Solar et al. \cite{solar2016scheduling} describe the proposal process in detail.

The observation period can be fully automated, interactive or manual. The goal of
this work is to propose a scheduler capable of automating the process and eliminating
the need for while leaving open the possibility of human intervention. To evaluate the 
effectiveness of the scheduler, we assume a data set with targets and observation
times and appraise the outcome by the overall duration.

\hyperref[point:objectiveFunctions]{Take a look at point 5 of Extra notes}

The optimisation goals of this study are
\begin{itemize}
    \item Minimisation of the inactivity periods.
    \item Maximisation of time assigned to observations.
    \item Maximisation of the quality of the data.
\end{itemize}

\marginnote{Could also add a \textit{future works} and \textit{summary} section.}[0cm]

In this study, we investigate how the time spent on collecting a defined set of
observations can be minimised. The quality of the data is treated as a constraint: If
the scintillation of a target is found to be poor on arrival, the observation is postponed. In cases of mediocre quality, the observation time is extended to compensate
by providing more data.

\section*{\cite{naghib2019framework} \citetitle*{naghib2019framework}}

\subsection*{LSST}

The Large Synoptic Survey Telescope (LSST) is a large,
ground-based optical survey that will image half of the sky
every few nights from Cerro Pachon in Northern Chile. LSST
comprises an 8.4 m primary mirror and a 3.2 gigapixel camera.
With a 9.6 \(deg^2\) field of view, it will visit each part of its
18,000 \(deg^2\) primary survey area about 1000 times over the
course of 10 yr. Each visit will likely comprise a 15 s pair of
exposures with a single-visit depth of about 24.5 mag (AB)
(in the six bands u, g, r, i, z, and y). The revolutionary role of
this telescope calls for no less than optimal operation.

There are four primary science drivers for the LSST project:
the characterization of dark energy through the multiple
cosmological probes (e.g., gravitational weak lensing, luminosity distances from Type Ia supernovae, and baryon acoustic
oscillations), mapping the 3D distribution of stars within our
Galaxy, a census of solar system objects within the solar
system, and a detailed study of the transient and variable
universe. Each of these objectives has a different set of
constraints and requirements on how the observations are made
(e.g., the cadence of the observations, the number of filters as a
function of time, the acceptable air-mass range for an
observation). 

\subsection*{Problem Defination and Constraints}


Earlier algorithmic approaches to the scheduling of groundbased telescopes 
are heavily based on observation proposals.
Proposals are handcrafted sequences of scripted astronomical
observations. They are generally tested only for feasibility
(e.g., that a set of fields were visible, or lie within a specified
air-mass range, or within a window in time), but not necessarily
for optimality.

More recently, the development of more expensive groundbased instruments with complex missions made it impossible to
rely solely on handcrafted proposals. The need for more
efficient use of the instrument’s time led to the development of decision-making algorithms to optimize their science output.
The scheduling at the single-visit level is referred to as optimal scheduling and it is
stated that the optimal scheduling requires reevaluating the
future sequence of observations once it is interrupted, but the
necessary extra computation is neither affordable nor fast
enough. However, in this paper we show that the scheduling in
the single-visit level, optimal scheduling, can be quickly
recovered after an interruption, if a memoryless framework is
used. Thus, the optimality does not necessarily need to be
sacrificed because of the limited computational resources.

To run a ground-based telescope with multiple science
objectives, such as LSST, the scheduler has to offer
\textit{controllability}, \textit{adjustability}, and \textit{recoverability}

\textbf{\textit{This one is incomplete. There is more to add.}}

\section*{\cite{luo2018cost} \citetitle*{luo2018cost}}
\textbf{\textit{This one is incomplete. There is more to add.}}

\section*{Extra notes}
\begin{enumerate}
    \item In a lot of papers, it was observed that the observation requests were weighted by a priority assigned to them by a committee or group.
    \item In astronomy, observing programmes can be broadly classified as targeted observations of known sources and systematic scans of areas of the sky, such as surveys that
    aim to map a large-scale structure or discover new objects.
    \item A 2010 survey by Mora and Solar \cite{mora2010survey} presents algorithms used by the Hubble
    Space Telescope, Very Large Telescope in the Chilean desert, the Subaru Telescope
    on Hawaii, Gemini Observatory, Stratospheric Observatory for Infrared Astronomy
    mounted on a Boeing 747, the Green Bank Telescope in West Virginia and the Atacama Large Millimetre/submillimetre Array (ALMA). They conclude that ‘None of
    the existing professional astronomical scheduling solutions provides a full automatic
    scheduler under changing conditions and for several telescopes, and none of the
    existing algorithms has solved the dynamic priorities problem for this case.
    \item A more recent survey by Colome et al. \cite{colome2012research} provides a concise listing of all algorithms employed for scheduling in different telescopes. An update was provided
    by Solar et al. \cite{solar2016scheduling} in 2016 with a review of heuristic-based approaches such as
    Simulated Annealing and Ant Colony Optimisation.
    \item \label{point:objectiveFunctions} Most approaches in the literature describe telescope control software with scheduling
    components without describing a clear objective function. Lopez-Casado \cite{lopez2016development}
    formulated an optimisation goal consisting of a user score and the
    percentage of observations executed (relative to the planned observations). Colome´
    et al. \cite{colome2010tjo} formulated a merit function as a weighted sum of observation conditions
    (height above the horizon, moon brightness and distance, best observation time,
    distance from previous target and proposal priority), environment conditions (photometric quality) and proposal history (to counterbalance proposal priorities by time
    already spent on the proposal). Colome et al.’s \cite{colome2014artificial} multiobjective approach considers
    minimising slew time, maximising priority, minimising downtime and minimising
    zenith angle which affects the image quality in optical telescopes.
\end{enumerate}

\printbibliography

\end{document}