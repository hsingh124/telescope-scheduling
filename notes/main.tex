\documentclass{article}
\usepackage[utf8]{inputenc}
\usepackage{marginnote}
\usepackage{biblatex}
\addbibresource{References.bib}
\usepackage{geometry}
\geometry{
    a4paper,
    total={170mm,257mm},
    left=10mm,
    top=20mm,
}
\setlength\parindent{0pt}
\setlength{\parskip}{0.5em}

\title{Research notes on Telescope Scheduling}
\author{Harpreet Singh}
\date{2021}

\begin{document}

\begin{titlepage}
\maketitle
\end{titlepage}

\section*{\citetitle*{naghib2019framework}}

\subsection*{LSST}
\marginnote{This is a margin note using the geometry package, set at 
3cm vertical offset to the line it is typeseted.}[1cm]

The Large Synoptic Survey Telescope (LSST) is a large,
ground-based optical survey that will image half of the sky
every few nights from Cerro Pachon in Northern Chile. LSST
comprises an 8.4 m primary mirror and a 3.2 gigapixel camera.
With a 9.6 \(deg^2\) field of view, it will visit each part of its
18,000 \(deg^2\) primary survey area about 1000 times over the
course of 10 yr. Each visit will likely comprise a 15 s pair of
exposures with a single-visit depth of about 24.5 mag (AB)
(in the six bands u, g, r, i, z, and y). The revolutionary role of
this telescope calls for no less than optimal operation.

There are four primary science drivers for the LSST project:
the characterization of dark energy through the multiple
cosmological probes (e.g., gravitational weak lensing, luminosity distances from Type Ia supernovae, and baryon acoustic
oscillations), mapping the 3D distribution of stars within our
Galaxy, a census of solar system objects within the solar
system, and a detailed study of the transient and variable
universe. Each of these objectives has a different set of
constraints and requirements on how the observations are made
(e.g., the cadence of the observations, the number of filters as a
function of time, the acceptable air-mass range for an
observation). 

\subsection*{Problem Defination and Constraints}


Earlier algorithmic approaches to the scheduling of groundbased telescopes 
are heavily based on observation proposals.
Proposals are handcrafted sequences of scripted astronomical
observations. They are generally tested only for feasibility
(e.g., that a set of fields were visible, or lie within a specified
air-mass range, or within a window in time), but not necessarily
for optimality.

More recently, the development of more expensive groundbased instruments with complex missions made it impossible to
rely solely on handcrafted proposals. The need for more
efficient use of the instrument’s time led to the development of decision-making algorithms to optimize their science output.
The scheduling at the single-visit level is referred to as optimal scheduling and it is
stated that the optimal scheduling requires reevaluating the
future sequence of observations once it is interrupted, but the
necessary extra computation is neither affordable nor fast
enough. However, in this paper we show that the scheduling in
the single-visit level, optimal scheduling, can be quickly
recovered after an interruption, if a memoryless framework is
used. Thus, the optimality does not necessarily need to be
sacrificed because of the limited computational resources.

To run a ground-based telescope with multiple science
objectives, such as LSST, the scheduler has to offer
\textit{controllability}, \textit{adjustability}, and \textit{recoverability}

\printbibliography

\end{document}